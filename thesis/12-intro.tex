\Introduction

  Именованная сущность - это слово или словосочетание обозначающее
  предмет или явления определенной категории. Примерами именованных сущностей
  являются имена людей, названия организаций и локаций.
  Задача распознавания именованных сущностей (Named Entity Recognition, NER)
  состоит в выделении и классификация именованных сущностей в тексте.
  В рамках конференции CoNLL 2003 проводилось соревнование
  для оценки качества методов распознавания именованных сущностей четырех типов
  на англоязычном корпусе \citep{tjong2003introduction}.
  Для решения задачи NER предлагалось много разных подходов \citep{nadeau2007survey}.
  В последнее время было показано, что методы на основе нейронных сетей показывают
  лучшие результаты для различных языков и корпусов, включая CoNLL 2003 \citep{DBLP:journals/corr/YangSC16}.

  Вместо большого количества вручную построенных признаков
  решающих определенную задачу, нейросетевые методы используют универсальные
  векторные представления слов \citep{mikolov2013distributed}.
  Согласно гипотезе о дистрибутивности, эти представления
  кодируют в себе смысл слов \citep{sahlgren2008distributional}.
  Это позволяет строить мультизадачные и языконезависимые
  архитектуры \citep{collobert2011natural, DBLP:journals/corr/YangSC16}.

  Несмотря на то, что использование универсальных векторных представлений
  получило в последнее время огромную популярность в силу своей эффективности
  и огромной экономии человеческих усилий, большой интерес все еще представляет
  исследование возможностей использования высокоуровневых признаков
  в качестве входных данных для нейросетей.
  Так, например, в работах \citep{xu2014rc, bian2014knowledge} описано использование
  морфологических, синтаксических и семантических признаков для построения
  более совершенных векторных представлений слов.

  Compreno -  это технология автоматического анализа текстов на естественном языке,
  в основе которой
  лежит многоуровневое лингвистическое описание, создававшееся профессиональными
  лингвистами в течение длительного времени \citep{anisimovich2012syntactic}.
  Помимо ручного описания Compreno использует для анализа большое количество
  информации, извлекаемых различными статистическими методами из текстовых корпусов.
  В Compreno реализована процедура семантико-синтаксического анализа текста,
  в результате которой любому предложению на естественном языке (английском или русском)
  ставится в соответствие семантико-синтаксическое дерево, моделирующее смысл предложения
  и содержащее грамматическую и семантическую информацию о каждом слове предложения.

  В данной работе исследована возможность использования семантико-синтаксического
  анализатора Compreno в качестве источника высокоуровневых признаков для задачи
  NER на корпусе CoNLL 2003 в рамках нейросетевого подхода.

  Статья организована следующим образом: в части 1 проведен обзор связанных работ.
  Выбранная нейросетевая модель и способы внедрения синтактико-семантических признаков описаны в части 2.
  В части 3 описаны проведенные эксперименты и программная реализация.

  Полученные результаты показывают повышение F1-меры почти на 1\% на корпусе CoNLL 2003
  при использовании синтактико-семантических признаков Compreno (87.49\% против 88.47\%).
  При этом затраты на их внедрение были минимальными - инженерия над признаками не проводилась.
